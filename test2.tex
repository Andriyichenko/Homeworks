\documentclass{article} % ドキュメントクラスを指定  
\usepackage[dvipdfmx]{graphicx}  
\usepackage{tikz} % 図を描くためのパッケージ  
\usepackage{enumerate}
\usepackage{tcolorbox}
\usepackage{amsmath}





\usetikzlibrary{positioning, intersections, calc, arrows.meta, math} % tikzのlibraryを読み込み  

\begin{document} % ドキュメント開始  
% \begin{center}  
% \begin{tikzpicture}[scale=1,samples=300]  
%   \draw[->,>=stealth,semithick] (-4,0)--(4,0) node[right]{$x$}; %x軸
%   \draw[->,>=stealth,semithick] (0,-3)--(0,3) node[left]{$y$}; %y軸
%   \draw (0,0) node[below left]{O}; %原点
%   \begin{scope} \clip (-4,-3) rectangle (4,3);
%     \draw[thick] plot(\x,{sin(\x r)});
%     \draw[thick] plot(\x,{pow(\x,2)});
%     \draw[thick] plot(\x,\x);
%   \end{scope}  
% \draw ({-sqrt(3)},3) node[above]{$y=x^2$};
% \draw (3,3) node[right]{$y=x$};
% \draw (4,{sin(4 r)}) node[below]{$y=\sin x$};

% \begin{tikzpicture}[scale=1]

%   \draw (-1,2) node[left]{A} -- (3,3) node[right]{B};
%   \draw (-1,2) -- (4,1) node[right]{C};
%   \draw (-1,2) -- (3,0) node[right]{D};
%   \draw (-1,2) -- (1,-2) node[right]{E};
  
% \end{tikzpicture} 
% \noindent\textgt{【例題】}\par
% 次の2次方程式を解け。\vspace{-8pt}
% \begin{enumerate}[(1)\;]
%   \item $x^2=1$
%   \item $x^2-4x+1=0$
% \end{enumerate}
% \noindent\textgt{(解答)}\par
% (1)\quad $x=\pm 1$ \qquad
% (2)\quad $x=2 \pm \sqrt{3}$

% \end{tikzpicture}
% \end{center}  
\noindent\textgt{【例題】}\par
次の2次方程式を解け。\vspace{-8pt}
\begin{enumerate}[(1)\;]
  \item $x^2=1$
  \item $x^2-4x+1=0$
\end{enumerate}
\noindent\textgt{(解答)}\par
(1)\quad $x=\pm 1$ \qquad
(2)\quad $x=2 \pm \sqrt{3}$

\begin{align*}
\mathbb{V}[\pi'\bm{\tilde{R}}]&=\mathbb{E}[(\pi'\bm{\tilde{R}}-\pi'\mathbb{E}[\tilde{R}])(\pi'\bm{\tilde{R}}-\pi'\mathbb{E}[\bm{\tilde{R}}])']\\
  
&=\pi'\mathbb{E}[(\bm{\tilde{R}}-\mathbb{E}[\bm{\tilde{R}}])(\bm{\tilde{R}}-\mathbb{E}[\bm{\tilde{R}}])'](\pi')'\\
  
&=\pi'\mathbb{E}[(\bm{\tilde{R}}-\mathbb{E}[\bm{\tilde{R}}])(\bm{\tilde{R}}-\mathbb{E}[\bm{\tilde{R}}])']\pi\\
  
  
  
  
  
\end{align*}


\end{document} % ドキュメント終了
