\documentclass[a4paper]{jsarticle}
\usepackage{amsmath,amsfonts,amssymb,amsthm}
\usepackage{mathtools}
\usepackage{xcolor}
\usepackage{otf}
\usepackage{xspace}
\usepackage{newpxtext}
\usepackage{empheq,cases}
\usepackage[absolute]{textpos}
\usepackage{qrcode}
% \usepackage{epstopdf} % For handling bounding box issue
\usepackage[dvipdfmx]{graphicx} % For including graphics
\usepackage{float}
% \renewcommand{\abstractname}{注意事項}
% \renewcommand{\proofname}{\textbf{証明}}
% \renewcommand{\qed}{\unskip\nobreak\quad\qedsymbol}
\newtagform{textbf}[\textbf]{[}{]}
\usetagform{textbf}
\newcommand*{\ie}{\textbf{\textit{i.e.}}\@\xspace}
\theoremstyle{definition}
\newtheorem{dfn}{Definition}
\newtheorem{prop}[dfn]{Proposition}
\newtheorem{lem}[dfn]{Lemma}
\newtheorem{thm}[dfn]{Theorem}
\newtheorem{cor}[dfn]{Corollary}
\newtheorem{rem}[dfn]{Remark}
\newtheorem{fact}[dfn]{Fact}
\renewcommand{\qedsymbol}{$\blacksquare$}
\usepackage{lipsum} % 用于生成示例文本
\usepackage{float} % 强制浮动
\usepackage{tikz} % 用于定位
\title{\vspace{-4cm}\S 2.2\quad Sigmoidal Function}
\author{伊 冉(Andre YI)}
\date{\today}
\begin{document}
\maketitle
%%%概要を出力したい人はこのように記述
\vspace{-0.4cm}
\begin{figure}[H] % 使用 figure 环境
  \centering
  \begin{tikzpicture}[remember picture, overlay]
      \node[anchor=north east] at (current page.north east) {
          \includegraphics[width=2cm]{pics/qr.png} % 替换为你的图片文件
      };
      % 添加说明文字在二维码的正下方
      \node[anchor=north east, yshift=-2cm] at (current page.north east) {デジタル版はここ};
  \end{tikzpicture}
  \label{fig:my_label}
\end{figure}
\begin{dfn}[シグモイド関数]
Logistic関数 $\sigma(x):\mathbb{R}\to [0,1] $ は、次のように定義される。
\begin{equation}
  \sigma(x) = \frac{1}{1 + e^{-x}} \quad (x \in \mathbb{R})\label{eq1}
\end{equation}
式\eqref{eq1}によって、$\displaystyle\lim_{x\to -\infty}\sigma(x)=0$,$\quad\displaystyle\lim_{x\to +\infty}\sigma(x)=1$となる。\\
上の条件を満たすような関数をシグモイド関数と言う。\\
\begin{figure}[h]
  \centering
  \begin{minipage}{0.43\columnwidth}
    \centering
    \includegraphics[width=\columnwidth]{pics/p1.2.png}
    \caption{シグモイド関数}
    \label{fig:p1}
  \end{minipage}
  \hspace{5mm}
  \begin{minipage}{0.43\columnwidth}
    \centering
    \includegraphics[width=\columnwidth]{pics/p1.1.png}
    \caption{シグモイド関数の1階微分}
    \label{fig:p2}
  \end{minipage}
\end{figure}
\begin{lem}
シグモイド関数が微分可能である
\begin{proof}
まず、関数の微分は次の定義によって与えられます:
\[
f'(x) = \lim_{h \to 0} \frac{f(x + h) - f(x)}{h}
\]
この定義を用いて、シグモイド関数の微分を計算すると、次のようになります:\\
まず、\(f(x + h)\)と\(f(x)\)を計算します:
\[
f(x + h) = \frac{1}{1 + e^{-(x + h)}} = \frac{1}{1 + e^{-x} e^{-h}} = \frac{1}{1 + \frac{e^{-x}}{e^{h}}}
\]
次に、差 \(f(x + h) - f(x)\)を計算します:
\[
f(x + h) - f(x) = \frac{1}{1 + e^{-x} e^{-h}} - \frac{1}{1 + e^{-x}}
\]
この差を通分します:
\[
= \frac{(1 + e^{-x}) - (1 + e^{-x} e^{-h})}{(1 + e^{-x} e^{-h})(1 + e^{-x})}
= \frac{e^{-x} - e^{-x} e^{-h}}{(1 + e^{-x} e^{-h})(1 + e^{-x})}
\]

次に、これを微分の定義に代入します:
\[
f'(x) = \lim_{h \to 0} \frac{e^{-x}(1 - e^{-h})}{h(1 + e^{-x} e^{-h})(1 + e^{-x})}
\]
ここで、\(\lim_{h \to 0} \frac{1 - e^{-h}}{h} = 1\)\footnote{関数 \(e^{-h}\)のテイラー展開は以下のようになります:
\[
e^{-h} = 1 - h + \frac{h^2}{2!} - \frac{h^3}{3!} + \cdots
\]

したがって、\(1 - e^{-h}\)を考えると:
\[
1 - e^{-h} = h - \frac{h^2}{2!} + \frac{h^3}{3!} - \cdots
\]

\[
\frac{1 - e^{-h}}{h} = \frac{h - \frac{h^2}{2!} + \frac{h^3}{3!} - \cdots}{h}
\]
これを整理すると:
\[
  \frac{1 - e^{-h}}{h}= 1 - \frac{h}{2!} + \frac{h^2}{3!} - \cdots
\]
\(h\)が0に近づくと、右辺の高次の項(\(-\frac{h}{2!} + \frac{h^2}{3!} - \cdots\))は全て0に近づきます。したがって、
\[
\lim_{h \to 0} \frac{1 - e^{-h}}{h} =\lim_{h \to 0} \left(1 - \frac{h}{2!} + \frac{h^2}{3!} - \cdots\right) = 1
\]
}であることを用います。このため、別の形に表しましょう:
\[
f'(x) = e^{-x} \lim_{h \to 0} \frac{1 - e^{-h}}{h} \cdot \frac{1}{(1 + e^{-x} e^{-h})(1 + e^{-x})}
\]
$\lim_{h \to 0} \frac{1 - e^{-h}}{h} = 1 \quad \text{なので、これは次のようになります:}$
\[
f'(x) = e^{-x} \cdot \frac{1}{(1 + e^{-x})(1 + e^{-x})} = e^{-x} \cdot \frac{1}{(1 + e^{-x})^2}
\]
さらに、\(f(x)\)を使って展開します:
\[
f(x) = \frac{1}{1 + e^{-x}} \Rightarrow 1 - f(x) = \frac{e^{-x}}{1 + e^{-x}}
\]
したがって、次のように書けます:
\[
e^{-x} = (1 + e^{-x}) \cdot (1 - f(x))
\]
これらを微分の結果に代入すると:
\[
f'(x) = \frac{(1 + e^{-x})(1 - f(x))}{(1 + e^{-x})^2} = f(x)(1 - f(x))
\]
したがって、シグモイド関数の微分は次のように表されます:
\[
f'(x) = f(x)(1 - f(x))
\]


\end{proof}
\end{lem}
\end{dfn}

\begin{rem}
  測度 $\mu$ は情報を評価するために使われる判定のシステムとして扱うことができることを思い出してください。
  したがって、$d\mu(x) = \mu(dx)$ は、$x$ にインプットされた情報の評価を表します。
  結果として、積分 $\int f(x) d\mu(x)$ は、判定のシステム $\mu$ の下での関数 $f(x)$ の評価を表します。
  
\end{rem}

\begin{dfn}[Discriminatory Function]
  この定義は、区別関数(Discriminatory Function)の概念を導入します。
  これは以下の特性によって特徴づけられます。
  すなわち、ニューロンの出力の評価 $\sigma(\mathbf{w}^T x + \theta)$ が、任意の入力 $x$ に対して、測度 $\mu$ の下で、任意の閾値 $\theta$ と任意の重み $\mathbf{w}$ に対して消失する場合、$\mu$ は消失しなければなりません。
  つまり、$\mu$ は無効な測度です。\\
  次に、測度$\mu \in M(I_n)$ とします。関数 $\sigma$ が測度 $\mu$ に対して区別的であるとは、次の条件を満たす場合に呼ばれます:

  \[
  \int_{I_n} \sigma(\mathbf{w}^T x + \theta)d\mu(x) = 0, \, \forall \mathbf{w} \in \mathbb{R}^n, \, \forall \theta \in \mathbb{R} \Rightarrow \mu = 0.
  \]
  ここで、関数 $\sigma$ は必ずしもLogistic関数である必要はなく、必要な特性を満たす任意の関数で構いません。
  $P_{\mathbf{w},\theta} = \{x; \mathbf{w}^T x + \theta = 0\}$ を法線ベクトル $\mathbf{w}$ と $(n+1)$ 次の切片 $\theta$ を持つ超平面(hyperplane)とします。また、開半空間(open half-space)を次のように定義します。
  \[
  \mathcal{H}_{\mathbf{w},\theta}^+ = \mathcal{H}_{\mathbf{w},\theta} = \{x; \mathbf{w}^T x + \theta > 0\}, \quad \mathcal{H}_{\mathbf{w},\theta}^- = \{x; \mathbf{w}^T x + \theta < 0\}
  \]
従って、$\mathbb{R}^n = \mathcal{H}_{\mathbf{w},\theta}^+ \cup \mathcal{H}_{\mathbf{w},\theta}^- \cup \mathcal{P}_{\mathbf{w},\theta}$ となる。
\end{dfn}

\begin{dfn}[Baire測度]
  この定義は、符号付きBaire測度の概念を使用します。
  Baire測度の概念を使用する主な理由は、コンパクト関数に対してよく分析できる(すべてのコンパクトの連続関数はBaire可測であり、
  Baire空間のコンパクトの連続関数は、任意の有限Baire測度に関して可積分です)。
\end{dfn}
次に、$I_n = [0, 1]^n = [0, 1] \times \cdots \times [0, 1]$ としたとき、$I_n$ 上の有限の符号付き正則Baire測度の空間を $M(I_n)$ とする.
\begin{lem}
 $\mu \in M(I_n)$ とします。もし $\mu$ がすべての超平面と $\mathbb{R}^n$ の開半空間で0になるなら、$\mu$ は0です。すなわち、以下が成り立つとき、

\[
\mu(P_{\mathbf{w},\theta}) = 0, \quad \mu(\mathcal{H}_{\mathbf{w},\theta}) = 0, \quad \forall \mathbf{w} \in \mathbb{R}^n, \, \theta \in \mathbb{R},
\]
そのとき $\mu = 0$ です。

\begin{proof} 
$\mathbf{w} \in \mathbb{R}^n$ を固定します。次のように定義された線形関数 $F : L^\infty(\mathbb{R}) \to \mathbb{R}$ を考えます。

\[
F(h) = \int_{I_n} h(\mathbf{w}^T x) d\mu(x),
\]
ここで $L^\infty(\mathbb{R})$ はほぼ至る所($\mathrm{almost everywhere}\iff$ a.e.)で有界な関数の空間を表します。\\
$L^{\infty}(\mathbb{R})$ の元 $h$ は、$\mathbb{R}$ 上の有界な関数であり、$||h||_\infty = \text{ess}\displaystyle\sup_{x \in \mathbb{R}} |h(x)|\footnote{ess supを本質的上限と呼ばれ、定義は以下のようになります。
\[
(X,\mathcal{F},\mu) \text{ を測度空間, } f:X \to \mathbb{R} \text{ を可測関数とする. このとき,}
\quad\text{ess } \sup f = \inf \{ a \in \mathbb{R} \mid \mu(f > a) = 0 \}\]}
<\infty$ で定義されます。\\
$h\leq ||h||_\infty$ により、
\[
|F(h)| = \left| \int_{I_n} h(\mathbf{w}^T x) d\mu(x) \right| \leq ||h||_\infty \left|\int_{I_n} d\mu(x)\right| =  |\mu(I_n)| \cdot ||h||_\infty 
\]
ここで $\mu$ が有限測度である。
次に $h = 1_{[\theta,\infty)}$ とします。すなわち、$h$ は区間 $[\theta,\infty)$ の指示関数\footnote{指示函数$1_{[\theta,\infty)}(x)$の定義は
\[
1_{[\theta,\infty)}(x) = 
\begin{cases}
1, & x \geq \theta \\

0, & x < \theta
\end{cases}
\]}であり、$\mathbf{w}^T x\geq\theta$。すると、

\begin{align}
  F(h) = \int_{I_n} h(\mathbf{w}^T x) d\mu(x) &= \int_{\{\mathbf{w}^T x \geq \theta\}} \mathbf{1} d\mu(x)  +\int_{\{\mathbf{w}^T x < \theta\}} 0d\mu(x) \notag\\
  &= \int_{\{\mathbf{w}^T x \geq \theta\}} \mathbf{1}d\mu(x) + 0 \notag\\
  &=\mu(\mathcal{P}_{\mathbf{w},\theta}\cup\mathcal{H}_{\mathbf{w},\theta}^+) \notag\\
  &= \mu(\mathcal{P}_{\mathbf{w},\theta}) + \mu(\mathcal{H}_{\mathbf{w},\theta}^+) \notag\\
  &= 0\notag
\end{align}
同様に$h = 1_{(\theta,\infty)}$を考えると、

\begin{align}
  F(h) = \int_{I_n} h(\mathbf{w}^T x) d\mu(x) &= \int_{\{\mathbf{w}^T x > \theta\}} d\mu(x)+\int_{\{\mathbf{w}^T x \leq \theta\}} d\mu(x)\notag \\
  &= \mu(\mathcal{H}_{\mathbf{w},-\theta}) \notag\\
  &= 0\notag
\end{align}
指示函数$1_{[a,b]}(x)$の定義は
$$
1_{[a,b]}(x) = 
\begin{cases}
1, & x \in [a,b] \\

0, & x \notin [a,b]
\end{cases}
$$
\textbf{閉区間の場合:}

\[
1_{[a,b]}(x) = 1_{[a,\infty)}(x) - 1_{(b,\infty)}(x).
\]
確認すると、

\begin{itemize}
    \item \textbf{ $x < a$ の時:}
    \begin{align*}
    1_{[a,\infty)}(x) &= 0, \\
    1_{(b,\infty)}(x) &= 0,\\
    1_{[a,\infty)}(x) - 1_{(b,\infty)}(x) &= 0, \quad \text{指示関数として定義。}
    \end{align*}

    \item \textbf{ $a \leq x \leq b$ の時:}
    \begin{align*}
    1_{[a,\infty)}(x) &= 1, \\
    1_{(b,\infty)}(x) &= 0,\\
    1_{[a,\infty)}(x) - 1_{(b,\infty)}(x) &= 1, \quad \text{指示関数として定義。}
    \end{align*}

    \item \textbf{ $x > b$ の時:}
    \begin{align*}
    1_{[a,\infty)}(x) &= 1, \\
    1_{(b,\infty)}(x) &= 1,\\
    1_{[a,\infty)}(x) - 1_{(b,\infty)}(x) &= 0, \quad \text{指示関数として定義。}
    \end{align*}
\end{itemize}

同様に、開区間について:

\[
1_{(a,b)}(x) = 1_{[a,\infty)}(x) - 1_{(b,\infty)}(x).
\]

したがって、$F$の線形性\footnote{Fの線形性とは、$F(\alpha f+\beta g)=\alpha F(f)+\beta F(g) \quad (\alpha ,\beta\in \mathbb{R})$}より、次のように書けます。
\[
F\left(\sum_{i=1}^N \alpha_i 1_{J_i}\right) = \sum_{i=1}^N \alpha_i F(1_{J_i}) = 0,
\]
ここで $\alpha_i \in \mathbb{R}$、$J_i$ は区間です。シンプル関数が $L^\infty(\mathbb{R})$ において稠密であるため、$F' = 0$ が成り立ちます。よって、任意の固定された $f \in L^\infty(\mathbb{R})$ に対して、$(s_n)_n$ が存在して $s_n \to f\quad (n\to\infty)$ となります。
$F$ が有界かつ連続\footnote{連続関数ならば、関数記号と極限が交換可能である.その証明は$\displaystyle\epsilon - \delta$論法を使えば、簡単に証明できる}であるので、以下が成り立ちます。

\[
F(f) = F\left(\lim_{n \to \infty} s_n\right) = \lim_{n \to \infty} F(s_n) = 0.
\]

次に、測度 $\mu$ のフーリエ変換を計算すると。

\[
\hat{\mu}(\mathbf{w}) = \int_{I_n} e^{i \mathbf{w}^T x} d\mu(x) = \int_{I_n} \cos(\mathbf{w}^T x) d\mu(x) + i \int_{I_n} \sin(\mathbf{w}^T x) d\mu(x).\footnote{オイラーの公式:$e^{i\theta}=\cos(\theta)+i\sin(\theta)$}
\]

$F(\cos(\cdot))=\int_{I_n} \cos(\mathbf{w}^T x) d\mu(x),F(\sin(\cdot))=\int_{I_n} \sin(\mathbf{w}^T x) d\mu(x)$を用いて、$cos$と$sin$関数は全部有界関数なので、以下が成り立ちます。

\[
F(\cos(\cdot)) + i F(\sin(\cdot)) = 0, \quad \forall \mathbf{w} \in \mathbb{R}^n,
\]

フーリエ変換の単射\footnote{$f: X \to Y が$\text{が単射 (injective) であるとは,}
$\forall x_1, x_2 \in X, \quad x_1 \neq x_2 \implies f(x_1) \neq f(x_2)$

\text{となること。対偶をとっていいかえると,}
$f(x_1) = f(x_2) \implies x_1 = x_2$}性質から、$\mu = 0$ が成り立つ。
\end{proof}

  
\end{lem}

\begin{prop}
任意の連続シグモイド関数は、$M(I_n)$ のすべての測度 $\mu$ に対して区別可能である。


\begin{proof}
$\mu \in M(I_n)$ を固定された測度とする。以下の条件を満たす連続シグモイド $\sigma$ を選ぶ:
\begin{equation}
  \int_{I_n} \sigma(\mathbf{w}^T x + \theta)d\mu(x) = 0, \quad \forall \mathbf{w} \in \mathbb{R}^n, \theta \in \mathbb{R}.\label{eq2}
\end{equation}
$\mu=0$を証明すればいい。まず、以下の関数を構成すると、
\[
\sigma_\lambda(x) = \sigma(\lambda(\mathbf{w}^T x + \theta) + \phi)
\]

シグモイド関数の定義により:

\[
\lim_{\lambda \to \infty} \sigma_\lambda(x) = \begin{cases} 
1, & \text{if } \mathbf{w}^T x + \theta > 0 \\
0, & \text{if } \mathbf{w}^T x + \theta < 0 \\
\sigma(\phi), & \text{if } \mathbf{w}^T x + \theta = 0
\end{cases}
\]

境界関数 $\gamma(x)$ を定義すると

\[
\gamma(x) = \begin{cases} 
1, & \text{if } x \in \mathcal{H}^+_{\mathbf{w},\theta} \\
0, & \text{if } x \in \mathcal{H}^-_{\mathbf{w},\theta} \\
\sigma(\phi), & \text{if } x \in \mathcal{P}_{\mathbf{w},\theta}
\end{cases}
\]

$\sigma_\lambda(x) \to \gamma(x)\quad (\lambda\to\infty)$ は $\mathbb{R}^n$ 上で各点収束し、また $\sigma$ はシグモイド関数であるため、値は有限区間にある。
有界収束定理より、極限と積分の交換ができる。よって、
\begin{align}
  \lim_{\lambda \to \infty} \int_{I_n} \sigma_\lambda(x) d\mu(x) &= \int_{I_n} \gamma(x) d\mu(x)\notag\\
  &= \int_{\mathcal{H}^+_{\mathbf{w},\theta}} \gamma(x) d\mu(x) + \int_{\mathcal{H}^-_{\mathbf{w},\theta}} \gamma(x) d\mu(x) + \int_{\mathcal{P}_{\mathbf{w},\theta}} \gamma(x) d\mu(x) \notag\\
  &= \mu(\mathcal{H}^+_{\mathbf{w},\theta}) + \sigma(\phi) \mu(\mathcal{P}_{\mathbf{w},\theta})\notag
\end{align}






等式\eqref{eq2}より、 $\int_{I_n} \sigma_\lambda(x) d\mu(x)=0,\forall \lambda $となるので、

\[
\mu(\mathcal{H}^+_{\mathbf{w},\theta}) + \sigma(\phi)\mu(\mathcal{P}_{\mathbf{w},\theta}) = 0
\]
$\phi \to +\infty$ とすると、$\sigma = \dfrac{1}{1+e^{-\phi}}\to 1$より、

\[
\mu(\mathcal{H}^+_{\mathbf{w},\theta}) + \mu(\mathcal{P}_{\mathbf{w},\theta}) = 0
\]
同様に $\phi \to -\infty$ とすると、$\sigma = \dfrac{1}{1+e^{-\phi}}\to 0$より、

\begin{equation}
  \mu(\mathcal{H}^+_{\mathbf{w},\theta}) = 0, \quad \forall \mathbf{w} \in \mathbb{R}^n, \theta \in \mathbb{R}\label{eq3}
\end{equation}
最後に、$\mathcal{H}^+_{\mathbf{w},\theta} = \mathcal{H}^-_{-\mathbf{w},-\theta}$ \footnote{$\mathcal{H}^-_{-\mathbf{w},-\theta}\coloneq \{-(\mathbf{w}^T x+\theta)<0\} =\{\mathbf{w}^T x+\theta >0\}=\mathcal{H}^+_{\mathbf{w},\theta}$}
であることから、式\eqref{eq3}により、$\mu$は$\mathbb{R}^n$における全ての半空間でなくなる。測度 $\mu$ は零測度、つまり $\mu = 0$ である。したがって、$\sigma$ は区別可能である。
\end{proof} 
\end{prop}

\end{document}